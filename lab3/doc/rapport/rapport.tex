\documentclass[a4paper,11pt]{article}
\pagestyle{headings}

\usepackage[utf8]{inputenc}
\usepackage{diagbox}
\usepackage[french]{babel}
\usepackage{graphicx}
\usepackage{float}
\usepackage{fullpage}
\usepackage{hyperref}
\usepackage{diagbox}
\usepackage{enumitem}
\usepackage[T1]{fontenc}
\usepackage[]{algorithm2e}
\graphicspath{{images/}}

\title{Reconnaissance de formes et apprentissage automatique Projet 3}
\author{Auriane Reverdell, Felix Hähnlein, Nicolas Violette, Romain Duléry}
\date{\today}

\setlength{\oddsidemargin}{0.2cm}
\setlength{\evensidemargin}{-0.7cm}
\setlength{\parindent}{30pt}
\setlength{\textwidth}{15cm}
\setlength{\textheight}{24cm}
\setlength{\topmargin}{-.5in}
\setlength{\parskip}{1ex}


\begin{document}

\maketitle
\vspace{1cm}

\section{Problématique et objectifs}

\section{Utilisation de notre propre réseau de neurones}
\subsection{Architecture utilisée}
\subsection{Entraînement et choix de paramètres}
\subsection{Evaluation des résultats}
\subsubsection{Evaluation par courbes ROC et Precision-Recall}
\subsubsection{Visualisation des filtres obtenus}

\section{Apprentissage par transfert : Réseaux de neurones préentraînés}
\subsection{Présentation de la méthode utilisée}
\subsection{Entraînement}
\subsection{Evaluation des résultats}

\section{Complétion de la base de données d'entrée}
\subsection{Nécessité de compléter la base de données}
\subsection{Méthodes de complétion}

\section{Annexes}

\end{document}
