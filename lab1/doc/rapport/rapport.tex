\documentclass[a4paper,11pt]{article}
\pagestyle{headings}

\usepackage[utf8]{inputenc}
\usepackage[french]{babel}
\usepackage{graphicx}
\usepackage[T1]{fontenc}
\graphicspath{{images/}}

\title{Rapport du projet 1 de reconnaissance faciale}
\author{Auriane Reverdell, Félix Hahnlein, Romain Duléry}
\date{\today}

\setlength{\oddsidemargin}{0.2cm}
\setlength{\evensidemargin}{-0.7cm}
\setlength{\parindent}{30pt}
\setlength{\textwidth}{15cm}
\setlength{\textheight}{24cm}
\setlength{\topmargin}{-.5in}
\setlength{\parskip}{1ex}

\begin{document}

\maketitle
\vspace{1cm}

\section{Résumé du travail réalisé}
\section{Analyse des résultats expérimentaux}
\subsection{Phase d'apprentissage}
visionnage 3D
- test variations paramètres : pas, BD d'entrainement, colorspace
\subsection{Reconnaissance faciale}
- taille ROI :
      soit petites, détection peau humaine
          - masque gaussien pas intéressant
      soit grandes, taille visage
          - masque gaussien à utiliser
\subsection{Clustering}
- Difficultés : pas possible de filtrer la taille car possible d'avoir un visage loin
                pas possible de filtrer par forme car possible d'avoir un visage de biais ou coupé
- Idée : utiliser le fait qu'un visage est d'une seule couleur
-> utiliser la moyenne des couleurs du visage
-> détection de plusieurs personnes

\end{document}
