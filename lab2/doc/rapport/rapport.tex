\documentclass[a4paper,11pt]{article}
\pagestyle{headings}

\usepackage[utf8]{inputenc}
\usepackage[french]{babel}
\usepackage{graphicx}
\usepackage{float}
\usepackage{fullpage}
\usepackage{diagbox}
\usepackage{enumitem}
\usepackage[T1]{fontenc}
\graphicspath{{images/}}

\title{Reconnaissance de formes et apprentissage automatique Projet 1}
\author{Auriane Reverdell, Felix Hähnlein, Nicolas Violette, Romain Duléry}
\date{\today}

\setlength{\oddsidemargin}{0.2cm}
\setlength{\evensidemargin}{-0.7cm}
\setlength{\parindent}{30pt}
\setlength{\textwidth}{15cm}
\setlength{\textheight}{24cm}
\setlength{\topmargin}{-.5in}
\setlength{\parskip}{1ex}

\begin{document}

\maketitle
\vspace{1cm}

\section{Résumé du travail réalisé}

\section{Analyse des résultats expérimentaux}

    \subsection{Phase d'apprentissage}

	\subsubsection{Travail réalisé}

	\subsubsection{Analyse de l'influence des paramètres}

    \subsection{Détection faciale}

	\subsubsection{Détection de visage sans post-traitement}

	\subsubsection{Détection de visage avec post-traitement}

	\subsubsection{Évaluation des détections de visages}

\section{Conclusion}

\section{Notes Auriane pour qu'elle s'en rappelle}
    
    \subsection{Choses à vérifier}
	
	\begin{itemize}
	    \item En quoi l'expression du visage fait varier la détection ? (sourire avec les yeux plissés, etc.)
	    \item Visages occultés ? Lunettes de soleil, main sur la bouche, mains qui cachent les
		yeux.
	\end{itemize}

    \subsection{Analyse}

	Singe 1 : visage détecté lorsque le minNeighbor is small, un rectangle quand minNeighbor = 3 et
    	scaleFactor = 1.01, si on baisse le minNeighbour le nombre de rectangles augmentent.

\end{document}
